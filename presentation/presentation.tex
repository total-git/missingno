\documentclass[hyperref={pdfpagelabels=false}]{beamer}
\usepackage[ngerman]{babel}
\usepackage{lmodern}
\usepackage[utf8]{inputenc}
\usepackage{tikz}
\usepackage{textpos}
\usepackage{dsfont}
\usepackage{wrapfig}
\mode<presentation> { \usetheme{Montpellier} }

\newcommand{\IQ}{\mathds{Q}}
\newcommand{\IN}{\mathds{N}}
\newcommand{\IZ}{\mathds{Z}}
\newcommand{\hmod}{\ \widehat\bmod\ }

\title{Lineararithmetik: Simplex-Algoritmus, Fourier-Motzkin Eliminierung und Omega-Test}
\author{Felix Reihl}
\subtitle{Vortrag im Seminar Automatisches Beweisen\\Sommersemester 2014}

%\logo{ \includegraphics[scale=.07]{LMU_Siegel.pdf}\vspace{250pt} }
\logo{\pgfputat{\pgfxy(-.5,8.05)}{\pgfbox[center,base]{\includegraphics[width=1.1cm]{LMU_Siegel.pdf}}}}
%\addtobeamertemplate{frametitle}{}{%
%	\begin{textblock*}{9cm}(9cm,-1.8cm) %(.85\textwidth,-1cm)
%	\includegraphics[height=2cm,width=2cm]{LMU_Siegel.pdf}
%\end{textblock*}}

%\beamerdefaultoverlayspecification{<+->}
\begin{document}
% ---
\begin{frame} \titlepage
\end{frame} 
% ---
\begin{frame}
	\tableofcontents
\end{frame} 
% ---
\section{Lineararithmetik}
\begin{frame}
	\frametitle{Syntax}
	\begin{table}[H]
		\begin{tabular}{rcl}
			Formel   & : & Formel $\land$ Formel $\mid$ (Formel) $\mid$ Atom \\
			Atom     & : & Summe Operator Summe \\
			Operator & : & $=$ $\mid$ $\leq$ $\mid$ $<$ \\
			Summe    & : & Term $\mid$ Term $+$ Term \\
			Term     & : & Variable $\mid$ konstante Variable $\mid$ Konstante
		\end{tabular}
	\end{table}
\end{frame} 
% ---
\section{Simplex}
\begin{frame}
	\frametitle{Constraints}
	Gleichungen
	\[ a_1 x_1 + \dotsc + a_n x_n = 0 \]
	und untere bzw.\ obere Schranken
	\[ u_i \leq x_i \leq o_i. \]
\end{frame}
% ---
\subsection{Vorarbeiten}
\begin{frame}
	\frametitle{Umformungen}
	\begin{block}{Beispiel}
		\[ x + y \geq 2, \quad 2x - y \geq 0, \quad -x + 2y \geq 1 \]
		kann zu
		\begin{align*}
			x + y - s_1 &= 0 & s_1 \geq 2 \\
			2x - y - s_2 &= 0 & s_2 \geq 0 \\
			-x + 2y - s_3 &= 0 & s_3 \geq 1
		\end{align*}
		umgeformt werden.
	\end{block}
\end{frame}
% ---
\begin{frame}
	\frametitle{Graphische Repräsentation als Halbräume und Hyperebenen}
	\begin{figure}
		%\includegraphics{hyper.png}
	\end{figure}
\end{frame}
% ---
\begin{frame}
	\frametitle{Darstellung als Matrix}
	\[
	\begin{pmatrix}
		1 & 1 & -1 & 0 & 0 \\
		2 & -1 & 0 & -1 & 0 \\
		-1 & 2 & 0 & 0 & -1
	\end{pmatrix}
	\]
	\mbox{}\\
	Erfüllbarkeit äquivalent zur Existenx eines $x$, sodass
	\[ Ax = 0 \text{\quad und \quad} \bigwedge_{i=1}^m u_i \leq s_i \leq o_i. \]
	Hinterer Teil der Matrix ist immer Dreiecksmatrix mit $-1$ auf der Diagonalen.
\end{frame}
% ---
\begin{frame}
	\frametitle{Tabellenform}
	\begin{center}
		\begin{tabular}{r|rr}
			& $x$ & $y$ \\ \hline
			$s_1$ & $1$ & $1$ \\
			$s_2$ & $2$ & $-1$ \\
			$s_3$ & $-1$ & $2$
		\end{tabular}
		\qquad
		\begin{tabular}{rrr}
			$2$ & $\leq$ & $s_1$ \\
			$0$ & $\leq$ & $s_2$ \\
			$1$ & $\leq$ & $s_3$
		\end{tabular}
		\[ \mathcal{B} = \{ s_1, s_2, s_3 \}, \quad \mathcal{N} = \{ x, y \} \]
	\end{center}
\end{frame}
% ---
\subsection{Algorithmus}
\begin{frame}
	\frametitle{Algorithmus}
	Zuweisung
	\[ \alpha : \mathcal{B} \cup \mathcal{N} \rightarrow \IQ \]
	% Anfangs alle \alpha(x_i) = 0
	Invarianten des Algorithmus:
	\begin{itemize}
		\item $Ax = 0$
		\item $\forall x_j \in \mathcal{N} . u_j \leq \alpha(x_j) \leq o_j$
	\end{itemize}
	Simplex prüft iterativ, ob eine Variable $\in \mathcal{B}$ (z.B.\ $s_1, s_2, s_3$) existiert, die ihre Schranke verletzt.
\end{frame}
% ---
\begin{frame}
	Sei nun o.B.d.A.\ $\alpha(x_i) > o_i$.
	
	Eine Gleichung nach $x_i$ umstellen:
	\[ x_i = \sum_{x_j \in \mathcal{N}} a_{i,j} x_j \]
	$\Rightarrow$ ein passendes $x_j$ verändern, sodass $x_i$ kleiner wird. % und seine Schranke nicht mehr verletzt.

	Wenn kein passendes $x_j$ existiert $\Rightarrow$ \emph{Unerfüllbar}.

	u.U. ist jetzt die Invariante $\forall x_j \in \mathcal{N} . u_j \leq \alpha(x_j) \leq o_j$ verletzt.
\end{frame}
% ---
\begin{frame}
	\frametitle{Pivot-Operation}
	$x_j$ soll um 
	\[ \Theta = \frac{o_i - \alpha(x_i)}{a_{i,j}} \]
	verringert bzw.\ erhöht werden, sodass es seine Schranke verletzt.

	$\Rightarrow$ Wir vertauschen $x_i$ und $x_j$ in der Tabelle:
	\begin{enumerate}
		\item Löse Zeile $i$ nach $x_j$ auf.
		\item Setze das Ergebnis in alle anderen Zeilen ein, sodass $x_j$ eliminiert wird.
	\end{enumerate}
\end{frame}
% ---
\begin{frame}
	\frametitle{Beispiel}
	\begin{center}
		\begin{tabular}{r|rr}
			& $x$ & $y$ \\ \hline
			$s_1$ & $1$ & $1$ \\
			$s_2$ & $2$ & $-1$ \\
			$s_3$ & $-1$ & $2$
		\end{tabular}
		\qquad
		\begin{tabular}{rrr}
			$2$ & $\leq$ & $s_1$ \\
			$0$ & $\leq$ & $s_2$ \\
			$1$ & $\leq$ & $s_3$.
		\end{tabular}
	\end{center}
	Die Anfangsbelegung $\alpha(x_i) = 0\ \forall i$ verletzt $2 \leq s_1$. $x$ ist passender Kandidat für Pivot-Operation.

	$\Rightarrow$ $s_1$ um $2$ erhöhen $\Rightarrow$ $x$ um $2$ erhöhen.

	Gleichung in Zeile von $s_1$ nach $x$ auflösen:
	\begin{align*}
		s_1 &= x + y \\
		x & = s_1 - y
	\end{align*}
\end{frame}
% ---
\begin{frame}
	\frametitle{Beispiel (cont)}
	\begin{center}
		\begin{tabular}{r|rr}
			& $s_1$ & $y$ \\ \hline
			$x$ & $1$ & $-1$ \\
			$s_2$ & $2$ & $-3$ \\
			$s_3$ & $-1$ & $3$
		\end{tabular}
		\qquad
		\begin{tabular}{rrr}
			$\alpha(x)$ & $=$ & $2$ \\
			$\alpha(y)$ & $=$ & $0$ \\
			$\alpha(s_1)$ & $=$ & $2$ \\
			$\alpha(s_2)$ & $=$ & $4$ \\
			$\alpha(s_3)$ & $=$ & $-2$
		\end{tabular}
	\end{center}
	Untere Schranke von $s_3$ verletzt; $s_3$ muss also um $3$ inkrementiert werden. $\Rightarrow$ Pivot-Operation mit $s_3$ und $y$.
	\begin{center}
		\begin{tabular}{r|rr}
			& $s_1$ & $s_3$ \\ \hline
			$x$ & $2/3$ & $-1/3$ \\
			$s_2$ & $1$ & $-3$ \\
			$y$ & $1/3$ & $1/3$
		\end{tabular}
		\qquad
		\begin{tabular}{rrr}
			$\alpha(x)$ & $=$ & $1$ \\
			$\alpha(y)$ & $=$ & $1$ \\
			$\alpha(s_1)$ & $=$ & $2$ \\
			$\alpha(s_2)$ & $=$ & $1$ \\
			$\alpha(s_3)$ & $=$ & $1$
		\end{tabular}
	\end{center}
	% Zuweisung x \mapsto 1, y \mapsto 1 erfüllt
	% Variablen anordnen, damit keine Endlosschleife auftritt
\end{frame}
% ---
\section{Fourier-Motzkin Variableneliminierung}
\begin{frame}
	\frametitle{Fourier-Motzkin}
	Konjunktion von linearen Constraints über $\mathds{R}$:
	\begin{align*}
		\sum_{j=1}^n a_{i,j} \cdot x_j &= b_i \\
		\sum_{j=1}^n a_{i,j} \cdot x_j &\leq b_i
	\end{align*}
	Erfüllbar?
\end{frame}
\subsection{Eliminierung von Variablen}
\begin{frame}
	\frametitle{Eliminierung von Variablen in Gleichungen}
	$x_j$ soll eliminiert werden $\Rightarrow$ Gleichung $i$ nach $x_j$ umstellen (vorausgesetzt, dass $a_{i,j} \neq 0$). O.B.d.A.\ sei diese Variable $x_n$.
	\[ \sum_{j=1}^n a_{i,j} \cdot x_j = b_i \]
	wird zu
	\[ x_n = \frac{b_i}{a_{i,n}} - \sum_{j=1}^{n-1} \frac{a_{i,j}}{a_{i,n}} \cdot x_j. \]
	$\Rightarrow$ In die anderen Gleichungen einsetzen und iterativ fortsetzen.
\end{frame}
% ---
\begin{frame}
	\frametitle{Eliminierung von Variablen in Ungleichungen}
	\begin{itemize}
		\item Eliminierung einer Variablen durch Projektion auf die übrigen Variablen
		\item Erzeugt u.U.\ neue Constraints
	\end{itemize}
	\[ \sum_{j=1}^n a_{i,j} \cdot x_j \leq b_i \]
	Summe aufspalten und umstellen:
	\[ a_{i,n} \cdot x_n \leq b_i - \sum_{j=1}^{n-1} a_{i,j} \cdot x_j \]
	Durch $a_{i,n}$ teilen $\Rightarrow$ untere oder obere Schranke $\beta_i$ für $x_n$.
	% Je nachdem, ob a_ij > 0 oder < 0
\end{frame}
% ---
\subtitle{Beschränkte und unbeschränkte Variablen}
\begin{frame}
	\frametitle{Beschränkte und unbeschränkte Variablen}
	\begin{itemize}
		\item Eine unbeschränkte Variable ist nicht von beiden Seiten beschränkt
		\item Kann zusammen mit allen Constraints in denen sie vorkommt entfernt werden
	\end{itemize}
	Im Folgenden sei $x_n$ also beschränkt:
	\begin{align*}
		\beta_u &\leq x_n \leq \beta_o \\
		\beta_u &\leq \beta_o
	\end{align*}
	Wenn zu $0 < b_k$ oder $0 \leq b_k$ vereinfachbar $\Rightarrow$ Widerspruch

	Sonst alle Constraints mit $x_n$ entfernen.
\end{frame}
% ---
\begin{frame}
	\frametitle{Beispiel}
	\begin{align*}
		x_1 - x_2 &\leq 0 \\
		x_1 - x_3 &\leq 0 \\
		-x_1 + x_2 + 2x_3 &\leq 0 \\
		-x_3 &\leq -1
	\end{align*}
	\begin{itemize}
		\item $x_1$ eliminieren: zwei obere Schranken ($x_1 \leq x_2$ und $x_1 \leq x_3$) und eine untere Schranke ($x_2 + 2x_3 \leq x_1$)
		\item Neue Constraints $2x_3 \leq 0$ und $x_1 \leq x_3$	
	\end{itemize}
\end{frame}
% ---
\begin{frame}
	\frametitle{Beispiel (cont)}
	\begin{align*}
		2x_3 &\leq 0 \\
		x_2 + x_3 &\leq 0 \\
		-x_3 &\leq -1
	\end{align*}
	$x_2$ hat keine untere Schranke mehr $\Rightarrow$ zweite Ungleichung entfernen

	Rest lässt sich zu $1 \leq 0$ zusammenfassen $\Rightarrow$ Unerfüllbar
	% exponentielles Wachstum mit Anzahl der Variablen
\end{frame}
% ---
\section{Omega-Test}
\begin{frame}
	\frametitle{Omega-Test}
	Erfüllbarkeit von linearen Constraints über ganzzahligen Variablen der Form
	\[ \sum_{i=1}^n a_i x_i = b \text{\quad oder\quad }  \sum_{i=1}^n a_i x_i \leq b. \]
\end{frame}
% ---
\subsection{Gleichungen}
\begin{frame}
	\frametitle{Gleichungen}
	Wenn ein $x_j$ mit $a_j = \pm 1$ existiert, direkt umstellen (o.B.d.A.\ sei diese Variable $x_n$) und in die anderen Gleichungen einsetzen.
	\[ x_n = \frac{b}{a_n} - \sum_{i=1}^{n-1} \frac{a_i}{a_n} x_i. \]
	Sonst kann nicht durch $a_j$ geteilt werden, da sonst Vorfaktoren nicht mehr ganzzahlig sind.

	Im Folgenden: Verfahren mit dem ein Vorfaktor betragsmäßig kontinuierlich verkleinert wird, bis er $\pm 1$ erreicht.
\end{frame}
% ---
\begin{frame}
	\frametitle{Symmetrisches Modulo}
	\begin{align*}
		a \hmod b &:= a - b \cdot \left\lfloor \frac{a}{b} + \frac{1}{2} \right\rfloor =\\
		&\phantom{:}= \begin{cases} {a \bmod b,} & \text{falls } {a \bmod b < b/2} \\ {(a \bmod b) - b,} & \text{sonst} \end{cases}
	\end{align*}
	Wir wollen eine Ersetzung für $x_n$ finden.

	Definiere $m := a_n + 1$; $\sigma$ sei eine neue Variable.
\end{frame}
% ---
\begin{frame}
	\begin{block}{Bemerkungen}
		\begin{itemize}
			\item $a_n \hmod m = -1$
			\item $a_i - (a_i \hmod m) = m \lfloor a_i/m + 1/2 \rfloor$, also durch $m$ teilbar.
		\end{itemize}
	\end{block}
	Wir führen nun das Constraint
	\[ \sum_{i=1}^n (a_i \hmod m) x_i = m \sigma + b \hmod m \]
	ein und teilen die Summe auf zu
	\begin{align*}
		(a_n \hmod m) x_n &= m \sigma + b \hmod m - \sum_{i=1}^{n-1} (a_i \hmod m) x_i \\
		x_n &= -m \sigma - b \hmod m + \sum_{i=1}^{n-1} (a_i \hmod m) x_i.
	\end{align*}
\end{frame}
% ---
\begin{frame}
	Einsetzen in die anderen Gleichungen:
	\[ -a_n m \sigma + \sum_{i=1}^{n-1} ( a_n ( a_i \hmod m ) ) x_i = b + a_n ( b \hmod m ) \]
	Unter Verwendung von $a_n = m - 1$:
	\begin{multline*}
		-a_n m \sigma + \sum_{i=1}^{n-1} ( a_i - ( a_i \hmod m ) + m ( a_i \hmod m ) )  x_i =\\
		= b - ( b \hmod m ) + m ( b \hmod m ).
	\end{multline*}
	Wegen Bemerkung 2 kann durch $m$ geteilt werden:
	\[ -a_n \sigma + \sum_{i=1}^{n-1} ( \lfloor a_i/m + 1/2 \rfloor +  ( a_i \hmod m ) ) x_i = \lfloor b/m + 1/2 \rfloor + ( b \hmod m ) \]
\end{frame}
% ---
\begin{frame}
	\frametitle{Folgerung}
	Für den neuen Vorfaktor von $x_i$ gilt:
	\[ | \lfloor a_i/m + 1/2 \rfloor + (a_i \hmod m) | \leq 5/6 | a_i | \]
	$\Rightarrow$ Der Vorfaktor von $x_i$ wird betragsmäßig in jedem Schritt kleiner und ist ganzzahlig, also wird irgendwann $\pm 1$ erreicht.
\end{frame}
% ---
\subsection{Ungleichungen}
\begin{frame}
	Variable zum Eliminieren Auswählen, dann die drei Unterprozeduren aufrufen:
	\begin{itemize}
		\item Real-Shadow
		\item Dark-Shadow
		\item Gray-Shadow
	\end{itemize}
\end{frame}
\begin{frame}
	\frametitle{Verfahren}
	\begin{center}
		%\includegraphics[scale=0.8]{abbildung_flowchart.png}
	\end{center}
\end{frame}
% ---
\begin{frame}
	\frametitle{Real-Shadow}
	\begin{itemize}
		\item Ähnlich Fourier-Motzkin: Variable auswählen und auf die restlichen Variablen projizieren.
		\item Falls am Ende keine ganzzahlige Lösung gefunden wird, ist das gesamte System unerfüllbar.
		\item Überabschätzung, also gilt der Umkehrschluss nicht.
	\end{itemize}
	$z$ soll eliminiert werden; $\beta \leq bz$ und $cz \leq \gamma$ seinen lineare Constraints mit $c,b \in \IN$ und $\beta,\gamma$ als lineare Restausdrücke.

	Multiplizieren mit $c$ bzw.\ $b$ ergibt
	\[ c\beta \leq cbz \text{\quad und \quad} cbz \leq b\gamma, \]
	also $c\beta \leq b\gamma$, wenn ein solches $z$ existiert.
\end{frame}
% ---
\begin{frame}
	\frametitle{Dark-Shadow}
	\begin{itemize}
		\item Unterabschätzung: Wenn keine Lösung gefunden wird, heißt es nicht, dass das System unerfüllbar ist.
		\item Wir suchen $z$, sodass $c \beta \leq cbz \leq b \gamma$, also
			\[ \exists z \in \IZ. \frac{\beta}{b} \leq z \leq \frac{\gamma}{c}. \]
		\item Annahme: Solch ein $z$ existiert nicht. Dann gilt
			\[ i < \frac{\beta}{b} \leq \frac{\gamma}{c} < i+1, \]
			wobei $i = \lfloor \beta/b \rfloor$.
	\end{itemize}
\end{frame}
% ---
\begin{frame}
	\frametitle{Dark-Shadow (cont)}
	\begin{itemize}
		\item Abstand von $\beta/b$ zu $i$ ist $\geq 1/b$
		\item Abstand von $\gamma/c$ zu $i+1$ ist $\geq 1/c$
		\item In Kombination ergibt sich (negiert)
			\[ b\gamma - c\beta \geq (c-1)(b-1) \]
		\item Falls diese Ungleichung wahr ist, war die Annahme falsch, dass es keine ganzzahlige Lösung gibt. Also existiert eine Lösung und es kann \emph{erfüllbar} zurückgegeben werden.
		\item Falls $c=1$ oder $b=1$, sind die beiden Ungleichungen aus Real-Shadow und Dark-Shadow identisch
	\end{itemize}
\end{frame}
% ---
\begin{frame}
	\frametitle{Gray-Shadow}
	\begin{itemize}
		\item Eine Lösung muss
			\[ c\beta \leq cbz \leq b\gamma \]
			erfüllen
		\item Außerdem muss 
			\[ b\gamma - c\beta \leq cb - c - b \]
			gelten, da sonst die Lösung schon in Dark-Shadow gefunden worden wäre
		\item Zusammengefasst:
			\[ \beta \leq bz \leq \beta + \frac{cb - c - b}{c} \]
		\item Innerhalb dieser Schranke probiert der Omega-Test nun mögliche Werte aus.
	\end{itemize}
\end{frame}
% ---
\subsection{Vorverarbeitung}
\begin{frame}
	\frametitle{Vorverarbeitung in allgemeinen linearen Systemen}
	\begin{itemize}
		\item Im Beispiel
			\[ x_1 + x_2 \leq 2, \quad x_1 \leq 1, \quad x_2 \leq 1 \]
			ist die erste Ungleichung redundant. % Wenn den Variablen ihre obere bzw. untere Schranke zugewiesen wird, ist die andere Ungleichung erfüllt
		\item Im Beispiel
			\[ 2x_1 + x_2 \leq 2, \quad x_2 \leq 4, \quad x_1 \leq 3 \]
			kann aus den ersten beiden Ungleichungen $x_1 \leq -1$ hergeleitet werden, sodass die Schranke $x_1 \leq 3$ verschärft werden kann.
	\end{itemize}
\end{frame}
% ---
\begin{frame}
	\frametitle{Vorverarbeitung in ganzzahligen linearen Systemen}
	\begin{itemize}
		\item Falls Vorfaktoren nicht ganzzahlig sondern rational sind, mit dem kleinsten gemeinsamen Nenner ganzzahlig machen.
		\item In gleicher Weise schon ganzzahlige Vorfaktoren mit dem größten gemeinsamen Teiler minimieren. % Wenn die rechte Seite einer Gleichung b nicht durch dadurch teilbar ist, kann schon unerfüllbar zurückgegeben werden.
		\item Ungleichungen mit $<$ oder $>$ zu nicht-strikten Ungleichungen mit $\leq$ bzw.\ $\geq$ machen, indem $1$ addiert oder subtrahiert wird.
		\item Spezifischere Umformungen für binäre Variablen
	\end{itemize}
\end{frame}
% ---
\section{Zusammenfassung}
\begin{frame}
	\frametitle{Zusammenfassung}
	\begin{itemize}
		\item Simplex-Algorithmus (reelle Zahlen) und Branch-and-Bound (ganze Zahlen) effizient, auch für große Anzahl von Constraints
		\item Fourier-Motzkin (reelle Zahlen) und Omega-Test (ganze Zahlen) in der Theorie weniger effizient, aber leicht zu implementieren und für kleine Anzahl an Constraints kein großer Geschwindigkeitsunterschied
	\end{itemize}
\end{frame}
% ---
\section{}
\begin{frame}
	\center
	Vielen Dank für Ihre Aufmerksamkeit!

	Fragen?
	\begin{block}{Quellen}
		\begin{thebibliography}{99}
		\bibitem{kroening}
			\textsc{D. Kroening and O. Strichman},
			Decision Procedures: An Algorithmic Point of View,
			Springer Publishing Company, Incorporated, 1 edition, 2008,
			Kapitel 5
		\end{thebibliography}
	\end{block}
\end{frame}

\end{document}
