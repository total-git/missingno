\documentclass[hyperref={pdfpagelabels=false}]{beamer}
\usepackage[ngerman]{babel}
\usepackage{lmodern}
\usepackage[utf8]{inputenc}
\usepackage{tikz}
\usepackage{textpos}
\usepackage{dsfont}
\usepackage{wrapfig}
\mode<presentation> { \usetheme{Montpellier} }

\newcommand{\IQ}{\mathds{Q}}
\newcommand{\IN}{\mathds{N}}
\newcommand{\IZ}{\mathds{Z}}
\newcommand{\hmod}{\ \widehat\bmod\ }

\title{Text-basiertes Adventure-Game}
\author{Eva Braß, Felix Reihl}
\subtitle{Projekt für die Vorlesung Fortgeschrittene Funktionale Programmierung\\Wintersemester 1014/15}

%\logo{ \includegraphics[scale=.07]{LMU_Siegel.pdf}\vspace{250pt} }
\logo{\pgfputat{\pgfxy(-.5,8.05)}{\pgfbox[center,base]{\includegraphics[width=1.1cm]{LMU_Siegel.pdf}}}}
%\addtobeamertemplate{frametitle}{}{%
%	\begin{textblock*}{9cm}(9cm,-1.8cm) %(.85\textwidth,-1cm)
%	\includegraphics[height=2cm,width=2cm]{LMU_Siegel.pdf}
%\end{textblock*}}

%\beamerdefaultoverlayspecification{<+->}
\begin{document}
% ---
\begin{frame} \titlepage
\end{frame} 
% ---
\begin{frame}
	\tableofcontents
\end{frame} 
% ---
\section{Das Game}
\begin{frame}
	\frametitle{Das Spielprinzip}
		\begin{itemize}
		\item Per Texteingabe gibt der Spieler Anweisungen.
		\item Alle Geschehnisse und Beschreibungen werden als Text ausgegeben.
		\item Beispiel: --ToDo--
		\end{itemize}
\end{frame} 
% ---
\section{Diskussion des Codes}
\begin{frame}
\end{frame}
% ---
\subsection{Verwendete Techniken}
\begin{frame}
		\begin{itemize}
		\item Yesod Framework
		\item Persistent
		\item Forms
		\item Monaden
		\end{itemize}
\end{frame}
% ---
\subsection{Interessante Probleme}
\begin{frame}
\end{frame}
% ---
\section{Genutzte Bibliotheken und Codevorlagen}
\begin{frame}
\end{frame}
% ---
\section{Verteilung der Aufgaben}
\begin{frame}
\end{frame}
% ---
\end{document}
